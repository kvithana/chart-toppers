% Options for packages loaded elsewhere
\PassOptionsToPackage{unicode}{hyperref}
\PassOptionsToPackage{hyphens}{url}
%
\documentclass[
]{article}
\usepackage{lmodern}
\usepackage{amssymb,amsmath}
\usepackage{ifxetex,ifluatex}
\ifnum 0\ifxetex 1\fi\ifluatex 1\fi=0 % if pdftex
  \usepackage[T1]{fontenc}
  \usepackage[utf8]{inputenc}
  \usepackage{textcomp} % provide euro and other symbols
\else % if luatex or xetex
  \usepackage{unicode-math}
  \defaultfontfeatures{Scale=MatchLowercase}
  \defaultfontfeatures[\rmfamily]{Ligatures=TeX,Scale=1}
\fi
% Use upquote if available, for straight quotes in verbatim environments
\IfFileExists{upquote.sty}{\usepackage{upquote}}{}
\IfFileExists{microtype.sty}{% use microtype if available
  \usepackage[]{microtype}
  \UseMicrotypeSet[protrusion]{basicmath} % disable protrusion for tt fonts
}{}
\makeatletter
\@ifundefined{KOMAClassName}{% if non-KOMA class
  \IfFileExists{parskip.sty}{%
    \usepackage{parskip}
  }{% else
    \setlength{\parindent}{0pt}
    \setlength{\parskip}{6pt plus 2pt minus 1pt}}
}{% if KOMA class
  \KOMAoptions{parskip=half}}
\makeatother
\usepackage{xcolor}
\IfFileExists{xurl.sty}{\usepackage{xurl}}{} % add URL line breaks if available
\IfFileExists{bookmark.sty}{\usepackage{bookmark}}{\usepackage{hyperref}}
\hypersetup{
  pdftitle={Cleaning},
  pdfauthor={Discove-R-Weekly},
  hidelinks,
  pdfcreator={LaTeX via pandoc}}
\urlstyle{same} % disable monospaced font for URLs
\usepackage[margin=1in]{geometry}
\usepackage{color}
\usepackage{fancyvrb}
\newcommand{\VerbBar}{|}
\newcommand{\VERB}{\Verb[commandchars=\\\{\}]}
\DefineVerbatimEnvironment{Highlighting}{Verbatim}{commandchars=\\\{\}}
% Add ',fontsize=\small' for more characters per line
\usepackage{framed}
\definecolor{shadecolor}{RGB}{248,248,248}
\newenvironment{Shaded}{\begin{snugshade}}{\end{snugshade}}
\newcommand{\AlertTok}[1]{\textcolor[rgb]{0.94,0.16,0.16}{#1}}
\newcommand{\AnnotationTok}[1]{\textcolor[rgb]{0.56,0.35,0.01}{\textbf{\textit{#1}}}}
\newcommand{\AttributeTok}[1]{\textcolor[rgb]{0.77,0.63,0.00}{#1}}
\newcommand{\BaseNTok}[1]{\textcolor[rgb]{0.00,0.00,0.81}{#1}}
\newcommand{\BuiltInTok}[1]{#1}
\newcommand{\CharTok}[1]{\textcolor[rgb]{0.31,0.60,0.02}{#1}}
\newcommand{\CommentTok}[1]{\textcolor[rgb]{0.56,0.35,0.01}{\textit{#1}}}
\newcommand{\CommentVarTok}[1]{\textcolor[rgb]{0.56,0.35,0.01}{\textbf{\textit{#1}}}}
\newcommand{\ConstantTok}[1]{\textcolor[rgb]{0.00,0.00,0.00}{#1}}
\newcommand{\ControlFlowTok}[1]{\textcolor[rgb]{0.13,0.29,0.53}{\textbf{#1}}}
\newcommand{\DataTypeTok}[1]{\textcolor[rgb]{0.13,0.29,0.53}{#1}}
\newcommand{\DecValTok}[1]{\textcolor[rgb]{0.00,0.00,0.81}{#1}}
\newcommand{\DocumentationTok}[1]{\textcolor[rgb]{0.56,0.35,0.01}{\textbf{\textit{#1}}}}
\newcommand{\ErrorTok}[1]{\textcolor[rgb]{0.64,0.00,0.00}{\textbf{#1}}}
\newcommand{\ExtensionTok}[1]{#1}
\newcommand{\FloatTok}[1]{\textcolor[rgb]{0.00,0.00,0.81}{#1}}
\newcommand{\FunctionTok}[1]{\textcolor[rgb]{0.00,0.00,0.00}{#1}}
\newcommand{\ImportTok}[1]{#1}
\newcommand{\InformationTok}[1]{\textcolor[rgb]{0.56,0.35,0.01}{\textbf{\textit{#1}}}}
\newcommand{\KeywordTok}[1]{\textcolor[rgb]{0.13,0.29,0.53}{\textbf{#1}}}
\newcommand{\NormalTok}[1]{#1}
\newcommand{\OperatorTok}[1]{\textcolor[rgb]{0.81,0.36,0.00}{\textbf{#1}}}
\newcommand{\OtherTok}[1]{\textcolor[rgb]{0.56,0.35,0.01}{#1}}
\newcommand{\PreprocessorTok}[1]{\textcolor[rgb]{0.56,0.35,0.01}{\textit{#1}}}
\newcommand{\RegionMarkerTok}[1]{#1}
\newcommand{\SpecialCharTok}[1]{\textcolor[rgb]{0.00,0.00,0.00}{#1}}
\newcommand{\SpecialStringTok}[1]{\textcolor[rgb]{0.31,0.60,0.02}{#1}}
\newcommand{\StringTok}[1]{\textcolor[rgb]{0.31,0.60,0.02}{#1}}
\newcommand{\VariableTok}[1]{\textcolor[rgb]{0.00,0.00,0.00}{#1}}
\newcommand{\VerbatimStringTok}[1]{\textcolor[rgb]{0.31,0.60,0.02}{#1}}
\newcommand{\WarningTok}[1]{\textcolor[rgb]{0.56,0.35,0.01}{\textbf{\textit{#1}}}}
\usepackage{graphicx,grffile}
\makeatletter
\def\maxwidth{\ifdim\Gin@nat@width>\linewidth\linewidth\else\Gin@nat@width\fi}
\def\maxheight{\ifdim\Gin@nat@height>\textheight\textheight\else\Gin@nat@height\fi}
\makeatother
% Scale images if necessary, so that they will not overflow the page
% margins by default, and it is still possible to overwrite the defaults
% using explicit options in \includegraphics[width, height, ...]{}
\setkeys{Gin}{width=\maxwidth,height=\maxheight,keepaspectratio}
% Set default figure placement to htbp
\makeatletter
\def\fps@figure{htbp}
\makeatother
\setlength{\emergencystretch}{3em} % prevent overfull lines
\providecommand{\tightlist}{%
  \setlength{\itemsep}{0pt}\setlength{\parskip}{0pt}}
\setcounter{secnumdepth}{-\maxdimen} % remove section numbering

\title{Cleaning}
\author{Discove-R-Weekly}
\date{10/15/2020}

\begin{document}
\maketitle

\begin{Shaded}
\begin{Highlighting}[]
\KeywordTok{library}\NormalTok{(knitr)}
\KeywordTok{library}\NormalTok{(tsibble)}
\KeywordTok{library}\NormalTok{(here)}
\KeywordTok{library}\NormalTok{(tidyverse)}
\end{Highlighting}
\end{Shaded}

\hypertarget{chart-toppers}{%
\section{Chart Toppers}\label{chart-toppers}}

\hypertarget{what-the-aria-charts-and-hottest-100-say-about-australias-changing-music-tastes.}{%
\subsection{What the ARIA charts and Hottest 100 say about Australia's
changing music
tastes.}\label{what-the-aria-charts-and-hottest-100-say-about-australias-changing-music-tastes.}}

Our team aims to investigate how Australian music tastes have changed
over time by comparing the audio features of tracks from the yearly ARIA
and Triple-J hottest 100 charts from 1970-2019. We will be using the
Spotify API to map the tracks from the charts to its corresponding
Spotify URI, allowing us to analyse individual tracks against key
attributes: duration, key, mode, time signature, acousticness,
speechiness, danceability, energy, loudness, valence/happiness and
tempo.

The documentation for the endpoint we are using to get the analysis data
is available
\href{https://developer.spotify.com/documentation/web-api/reference/tracks/get-audio-features/}{here}.
We are sourcing the ARIA charts and Triple J Hottest 100 rankings from
these sites: - \url{https://www.aria.com.au/charts/2019/singles-chart} -
\url{https://en.wikipedia.org/wiki/Triple_J_Hottest_100}

\hypertarget{data-sourcing}{%
\subsection{Data Sourcing}\label{data-sourcing}}

The data was sourced through calling the above endpoint of the Spotify
API. This was done by using their Javascript client libraries since it
was what we were most familiar with.

The source code for pulling data from Spotify is available on Github
\href{https://github.com/kvithana/spotify-audio-features-to-csv}{here}.

We created or sourced a Spotify playlists which included the songs for
each year's charts from 1970 - 2019. This playlist could then be passed
to the CLI tool.

Given a list of playlist URIs or URLs in a \texttt{.txt} file, the
program will get the tracks associated with that playlist, and call the
\href{https://developer.spotify.com/documentation/web-api/reference/tracks/get-audio-features/}{Spotify
Audio Features} endpoint to get insightful information about the tracks.

This data was outputted in csv format into the following structure:

\begin{verbatim}
└───data
    │   playlists.csv
    │
    ├───features
    │       0hIiy3ihpzsIX9Dd6RVtWw.csv
    │       0kgHtoYJSMS3pMMciC3Us4.csv
    │       ...
    │
    └───tracks
            0hIiy3ihpzsIX9Dd6RVtWw.csv
            0kgHtoYJSMS3pMMciC3Us4.csv
            ...
\end{verbatim}

\begin{itemize}
\tightlist
\item
  \texttt{playlists.csv} : a lookup of the playlist URI and the playlist
  name and description
\item
  \texttt{/features}: folder with an individual csv for each playlist,
  containing the audio features data.
\item
  \texttt{/tracks}: folder with an individual csv for each playlist,
  containing data about each track.
\end{itemize}

\hypertarget{playlists.csv}{%
\subsubsection{\texorpdfstring{\texttt{playlists.csv}}{playlists.csv}}\label{playlists.csv}}

Has consolidated information about each playlist.

Primary Key:

\begin{itemize}
\tightlist
\item
  \texttt{playlist\_uri}: Spotify URI for the playlist. Also the title
  of corresponding CSV files.
\end{itemize}

Other Attributes:

\begin{itemize}
\tightlist
\item
  \texttt{title}: Title of the playlist
\item
  \texttt{description}: Description of the playlist
\item
  \texttt{url}: URL of the playlist
\end{itemize}

After cleaning \& consolidation: - \texttt{year}: Year of the chart -
\texttt{chart}: \texttt{H100} or \texttt{ARIA}

\hypertarget{features...}{%
\subsubsection{\texorpdfstring{\texttt{features/...}}{features/...}}\label{features...}}

Has audio features for each track returned from the
\href{https://developer.spotify.com/documentation/web-api/reference/tracks/get-audio-features/}{Spotify
Audio Features} endpoint.

\emph{Primary Keys:} - \texttt{playlist\_uri}: Spotify URI for the
playlist - \texttt{uri}: Spotify URI of the track

\emph{Other Attributes:} The following features are saved in the CSV.
Read the description and distribution for each attribute from the above
Spotify API documentation.

\begin{itemize}
\tightlist
\item
  \texttt{duration\_ms}
\item
  \texttt{key}
\item
  \texttt{mode}
\item
  \texttt{time\_signature}
\item
  \texttt{acousticness}
\item
  \texttt{danceability}
\item
  \texttt{energy}
\item
  \texttt{instrumentalness}
\item
  \texttt{liveness}
\item
  \texttt{loudness}
\item
  \texttt{speechiness}
\item
  \texttt{valence}
\item
  \texttt{tempo}
\end{itemize}

\hypertarget{tracks...}{%
\subsubsection{\texorpdfstring{\texttt{tracks/...}}{tracks/...}}\label{tracks...}}

Idenitifying information about each track, as returned from the
\href{https://developer.spotify.com/documentation/web-api/reference/tracks/get-several-tracks/}{Spotify
Get Tracks} endpoint.

\emph{Note: Tracks can have multiple artists. The CSV has been formatted
to have an entry for each listed artist of the track.}

\emph{Primary Keys:}

\begin{itemize}
\tightlist
\item
  \texttt{playlist\_uri}: Spotify URI for the playlist
\item
  \texttt{uri}: Spotify URI of the track
\item
  \texttt{artist}: name of a featuring artist
\item
  \texttt{artist\_uri}: Spotify URI of the artist
\end{itemize}

\emph{Other Attributes:}

The following features are saved in the CSV. Read the description and
distribution for each attribute from the above Spotify API
documentation.

\begin{itemize}
\tightlist
\item
  \texttt{album}: name of the album
\item
  \texttt{album\_uri}: Spotify URI for the album
\item
  \texttt{disc\_number}
\item
  \texttt{duration\_ms}
\item
  \texttt{name}
\item
  \texttt{popularity}
\item
  \texttt{explicit}
\item
  \texttt{uri}
\item
  \texttt{link}: renamed from \texttt{href}
\end{itemize}

\hypertarget{data-cleaning-consolidation}{%
\subsection{Data Cleaning \&
Consolidation}\label{data-cleaning-consolidation}}

After the above process, we had to get the data into R and create our
dataframes. Since the data coming from the Spotify API is reliable and
clean, most of the work involves merging together the data from the
separate csvs into a single dataframe. Our process is outlined below.

\hypertarget{import-csv}{%
\subsubsection{Import CSV}\label{import-csv}}

First, let's get the \texttt{playlists.csv} file which has information
about all the playlists included in the dataset.

\begin{Shaded}
\begin{Highlighting}[]
\NormalTok{playlist_data <-}\StringTok{ }\KeywordTok{read_csv}\NormalTok{(here}\OperatorTok{::}\KeywordTok{here}\NormalTok{(}\StringTok{'Data/playlists.csv'}\NormalTok{))}

\NormalTok{features_data <-}
\StringTok{    }\KeywordTok{list.files}\NormalTok{(}\DataTypeTok{path =}\NormalTok{ here}\OperatorTok{::}\KeywordTok{here}\NormalTok{(}\StringTok{'Data/features'}\NormalTok{),}
               \DataTypeTok{pattern =} \StringTok{"*.csv"}\NormalTok{, }
               \DataTypeTok{full.names =}\NormalTok{ T) }\OperatorTok\StringTok{ }
\StringTok{    }\KeywordTok{map_df}\NormalTok{(}\OperatorTok{~}\KeywordTok{read_csv}\NormalTok{(.))}

\NormalTok{tracks_data <-}\StringTok{ }
\StringTok{    }\KeywordTok{list.files}\NormalTok{(}\DataTypeTok{path =}\NormalTok{ here}\OperatorTok{::}\KeywordTok{here}\NormalTok{(}\StringTok{'Data/tracks'}\NormalTok{),}
               \DataTypeTok{pattern =} \StringTok{"*.csv"}\NormalTok{, }
               \DataTypeTok{full.names =}\NormalTok{ T) }\OperatorTok\StringTok{ }
\StringTok{    }\KeywordTok{map_df}\NormalTok{(}\OperatorTok{~}\KeywordTok{read_csv}\NormalTok{(.))}

\KeywordTok{glimpse}\NormalTok{(playlist_data)}
\end{Highlighting}
\end{Shaded}

\begin{verbatim}
## Rows: 77
## Columns: 4
## $ playlist_uri <chr> "spotify:playlist:6YKI2VYSO9iZtyaLb3ZUsG", "spotify:pl...
## $ title        <chr> "ARIA Top 100 Singles of 1970", "ARIA Top 100 Singles ...
## $ description  <chr> "Unavailable: Lionel Rose - I Thank You, Mary Hopkin -...
## $ url          <chr> "https://api.spotify.com/v1/playlists/6YKI2VYSO9iZtyaL...
\end{verbatim}

\begin{Shaded}
\begin{Highlighting}[]
\KeywordTok{glimpse}\NormalTok{(features_data)}
\end{Highlighting}
\end{Shaded}

\begin{verbatim}
## Rows: 7,605
## Columns: 15
## $ playlist_uri     <chr> "spotify:playlist:03vMytYzGsKy1dQkGDdiCp", "spotif...
## $ duration_ms      <dbl> 269667, 180566, 229526, 241693, 295502, 176561, 25...
## $ key              <dbl> 0, 4, 10, 4, 5, 7, 5, 1, 5, 2, 1, 9, 1, 9, 2, 0, 4...
## $ mode             <dbl> 1, 1, 1, 0, 0, 0, 0, 1, 0, 1, 1, 0, 1, 1, 1, 0, 0,...
## $ time_signature   <dbl> 4, 4, 4, 4, 4, 4, 4, 4, 4, 4, 4, 4, 4, 4, 4, 4, 4,...
## $ acousticness     <dbl> 0.00801, 0.16600, 0.36900, 0.63400, 0.32900, 0.003...
## $ danceability     <dbl> 0.856, 0.782, 0.689, 0.566, 0.470, 0.723, 0.489, 0...
## $ energy           <dbl> 0.609, 0.685, 0.481, 0.664, 0.431, 0.809, 0.597, 0...
## $ instrumentalness <dbl> 8.15e-05, 1.18e-05, 1.03e-06, 0.00e+00, 0.00e+00, ...
## $ liveness         <dbl> 0.0344, 0.1600, 0.0649, 0.1160, 0.0854, 0.5650, 0....
## $ loudness         <dbl> -7.223, -6.237, -7.503, -5.303, -6.129, -3.081, -6...
## $ speechiness      <dbl> 0.0824, 0.0309, 0.0815, 0.0464, 0.0342, 0.0625, 0....
## $ valence          <dbl> 0.928, 0.603, 0.283, 0.437, 0.289, 0.274, 0.324, 0...
## $ tempo            <dbl> 114.988, 118.016, 80.025, 128.945, 157.980, 98.007...
## $ uri              <chr> "spotify:track:32OlwWuMpZ6b0aN2RZOeMS", "spotify:t...
\end{verbatim}

\begin{Shaded}
\begin{Highlighting}[]
\KeywordTok{glimpse}\NormalTok{(tracks_data)}
\end{Highlighting}
\end{Shaded}

\begin{verbatim}
## Rows: 8,893
## Columns: 12
## $ playlist_uri <chr> "spotify:playlist:03vMytYzGsKy1dQkGDdiCp", "spotify:pl...
## $ album        <chr> "Uptown Special", "Uptown Special", "Me 4 U", "Me 4 U"...
## $ album_uri    <chr> "spotify:album:3vLaOYCNCzngDf8QdBg2V1", "spotify:album...
## $ artist       <chr> "Mark Ronson", "Bruno Mars", "OMI", "Felix Jaehn", "Wi...
## $ artist_uri   <chr> "spotify:artist:3hv9jJF3adDNsBSIQDqcjp", "spotify:arti...
## $ disc_number  <dbl> 1, 1, 1, 1, 1, 1, 1, 1, 1, 1, 1, 1, 1, 1, 1, 1, 1, 1, ...
## $ duration_ms  <dbl> 269666, 269666, 180565, 180565, 229525, 229525, 241693...
## $ name         <chr> "Uptown Funk (feat. Bruno Mars)", "Uptown Funk (feat. ...
## $ popularity   <dbl> 81, 81, 76, 76, 82, 82, 74, 73, 60, 60, 60, 71, 80, 77...
## $ explicit     <lgl> FALSE, FALSE, FALSE, FALSE, FALSE, FALSE, FALSE, FALSE...
## $ uri          <chr> "spotify:track:32OlwWuMpZ6b0aN2RZOeMS", "spotify:track...
## $ link         <chr> "https://api.spotify.com/v1/tracks/32OlwWuMpZ6b0aN2RZO...
\end{verbatim}

\hypertarget{extracting-info-from-playlist-titles}{%
\subsection{Extracting Info From Playlist
Titles}\label{extracting-info-from-playlist-titles}}

We need to extract the year of the playlist and also whether it was for
the \texttt{Hottest\ 100} of \texttt{ARIA\ Chart}. We can give these the
flags \texttt{H100} and \texttt{ARIA}.

\begin{Shaded}
\begin{Highlighting}[]
\NormalTok{playlist_data <-}\StringTok{ }\NormalTok{playlist_data }\OperatorTok
\StringTok{  }\KeywordTok{mutate}\NormalTok{(}\DataTypeTok{year =} \KeywordTok{as.numeric}\NormalTok{(}\KeywordTok{str_extract}\NormalTok{(title, }\StringTok{"}\CharTok{\textbackslash{}\textbackslash{}}\StringTok{d\{4\}$"}\NormalTok{))) }\OperatorTok
\StringTok{  }\KeywordTok{mutate}\NormalTok{(}\DataTypeTok{chart =} \KeywordTok{as.factor}\NormalTok{(}
    \KeywordTok{if_else}\NormalTok{(}
      \KeywordTok{grepl}\NormalTok{(}\StringTok{"ARIA"}\NormalTok{, title, }\DataTypeTok{fixed =} \OtherTok{TRUE}\NormalTok{),}
      \StringTok{"ARIA"}\NormalTok{,}
      \StringTok{"H100"}
\NormalTok{      )}
\NormalTok{    )}
\NormalTok{  )}

\KeywordTok{head}\NormalTok{(playlist_data)}
\end{Highlighting}
\end{Shaded}

\begin{verbatim}
## # A tibble: 6 x 6
##   playlist_uri      title      description            url             year chart
##   <chr>             <chr>      <chr>                  <chr>          <dbl> <fct>
## 1 spotify:playlist~ ARIA Top ~ Unavailable: Lionel R~ https://api.s~  1970 ARIA 
## 2 spotify:playlist~ ARIA Top ~ Unavailable: Lally St~ https://api.s~  1971 ARIA 
## 3 spotify:playlist~ ARIA Top ~ Unavailable: Slim New~ https://api.s~  1972 ARIA 
## 4 spotify:playlist~ ARIA Top ~ Unavailable: Jamie Re~ https://api.s~  1973 ARIA 
## 5 spotify:playlist~ ARIA Top ~ Unavailable: Sister J~ https://api.s~  1974 ARIA 
## 6 spotify:playlist~ ARIA Top ~ Unavailable: Bob Huds~ https://api.s~  1975 ARIA
\end{verbatim}

\hypertarget{analysing-data-distribution}{%
\subsection{Analysing Data
Distribution}\label{analysing-data-distribution}}

Let's ensure we have enough data by graphing the number of tracks in
each chart for each year. Note that the Hottest 100 charts begin in
1989. Some tracks are missing in the dataset as they are not available
on Spotify. This is only a small percentage of tracks.

\begin{Shaded}
\begin{Highlighting}[]
\NormalTok{features_data }\OperatorTok\StringTok{ }
\StringTok{  }\KeywordTok{left_join}\NormalTok{(playlist_data, }\DataTypeTok{by =} \StringTok{"playlist_uri"}\NormalTok{) }\OperatorTok
\StringTok{  }\KeywordTok{group_by}\NormalTok{(year, chart) }\OperatorTok
\StringTok{  }\KeywordTok{tally}\NormalTok{(}\DataTypeTok{name =} \StringTok{"count"}\NormalTok{) }\OperatorTok
\StringTok{  }\KeywordTok{ggplot}\NormalTok{(}\KeywordTok{aes}\NormalTok{(}\DataTypeTok{fill=}\NormalTok{chart, }\DataTypeTok{y=}\NormalTok{count, }\DataTypeTok{x=}\NormalTok{year)) }\OperatorTok{+}\StringTok{ }
\StringTok{    }\KeywordTok{geom_bar}\NormalTok{(}\DataTypeTok{position=}\StringTok{"dodge"}\NormalTok{, }\DataTypeTok{stat=}\StringTok{"identity"}\NormalTok{) }\OperatorTok{+}
\StringTok{  }\KeywordTok{facet_wrap}\NormalTok{(}\OperatorTok{~}\NormalTok{chart)}
\end{Highlighting}
\end{Shaded}

\includegraphics{Chart-Toppers-Data_files/figure-latex/graph-data-1.pdf}

\hypertarget{write-to-csv}{%
\subsection{Write To CSV}\label{write-to-csv}}

We can now write the consolidated dataframes to csvs for submission.

\begin{Shaded}
\begin{Highlighting}[]
\KeywordTok{write.csv}\NormalTok{(playlist_data,}\DataTypeTok{file =}\NormalTok{ here}\OperatorTok{::}\KeywordTok{here}\NormalTok{(}\StringTok{'/Data/cleaned/playlists.csv'}\NormalTok{))}
\KeywordTok{write.csv}\NormalTok{(features_data,}\DataTypeTok{file =}\NormalTok{ here}\OperatorTok{::}\KeywordTok{here}\NormalTok{(}\StringTok{'/Data/cleaned/features.csv'}\NormalTok{))}
\KeywordTok{write.csv}\NormalTok{(tracks_data,}\DataTypeTok{file =}\NormalTok{ here}\OperatorTok{::}\KeywordTok{here}\NormalTok{(}\StringTok{'/Data/cleaned/tracks.csv'}\NormalTok{))}
\end{Highlighting}
\end{Shaded}

\end{document}
